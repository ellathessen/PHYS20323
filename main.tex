\documentclass[11.5pt]{article}
\usepackage{epsfig}
\usepackage{times}
\usepackage{fancyhdr}
\usepackage{pslatex}
\usepackage{amsmath}
\usepackage{mathrsfs}
\usepackage[dvipsnames]{xcolor}
\usepackage[hidelinks]{hyperref}%renewcommand{\topfraction}{1.0}
\renewcommand{\topfraction}{1.0}
\renewcommand{\bottomfraction}{1.0}
\renewcommand{\textfraction}{0.0}
\setlength {\textwidth}{6.6in}
\hoffset=-1.0in
\oddsidemargin=1.00in
\marginparsep=0.0in
\marginparwidth=0.0in                                                                               
\setlength {\textheight}{9.0in}
\voffset=-1.00in
\topmargin=1.0in
\headheight=0.0in
\headsep=0.00in
\footskip=0.50in                                         
\setcounter{page}{1}
\begin{document}
\def\pos{\medskip\quad}
\def\subpos{\smallskip \qquad}
\newfont{\nice}{cmr12 scaled 1250}
\newfont{\name}{cmr12 scaled 1080}
\newfont{\swell}{cmbx12 scaled 800}

\title{PHYS 20323/60323: Fall 2024}
\author{Ella Thessen}
\date{Friday December 6th}

\begin{center}
    {\Large
    PHYS 20323/60323: Fall 2024 - LaTeX by: Ella June Thessen:D}\vspace{.25cm}

\begin{enumerate}
    \item 
\text An electron is found to be in the spin state (in the $z$-basis) $\chi = A
\begin{pmatrix}
3i \\
4
\end{pmatrix}
$
\begin{enumerate}
    \item[(a)] (5 points) Determine the possible values of \( A \) such that the state is normalized.\vspace{1cm}
   \item[(b)] (5 points) Find the expectation values of the operators \color{red}\( S_x \),\color{purple} \( S_y \), \color{black}\( S_z \), and \( \vec{S}^2 \).\vspace{1cm}
\end{enumerate}
The matrix representations in the \( z \)-basis for the components of electron spin operators are given by:
\[\color{red}S_x = \frac{\hbar}{2} 
\begin{pmatrix}
0 & 1 \\ 
1 & 0
\end{pmatrix};\ \quad
\color{purple}S_y = \frac{\hbar}{2} 
\begin{pmatrix}
0 & -i \\ 
i & 0
\end{pmatrix}; \quad
\color{orange}S_z = \frac{\hbar}{2} 
\begin{pmatrix}
1 & 0 \\ 
0 & -1
\end{pmatrix}\vspace{0.25cm}
\] 
\textbf{}{2. The average electrostatic field in the earth’s atmosphere in fair weather is approximately given by:} 
\[
\vec{E} = E_0 
\left(
A e^{-\alpha z} + B e^{-\beta z}
\right)
\hat{z},
                                                                               (1\]
where \( A, B, \alpha, \beta \) are positive constants and \( z \) is the height above the (locally flat) earth surface.\vspace{.10cm}
\begin{enumerate}
    \item[(a)] (5 points) Find the average charge density in the atmosphere as a function of height.\vspace{0.25cm}
    \item[(b)] (5 points) Find the electric potential as a function of height above the earth.\vspace{0.25cm}  
\end{enumerate}

\textbf{3. The following questions refer to stars in the table below.} 

\vspace{.10cm}
Note: there may be many answers
\begin{table}[h!]
\centering
\begin{tabular}{|l|c|c|c|c|c|}
\hline
\textbf{Name} & \textbf{Mass} & \textbf{Luminosity} & \textbf{Lifetime } & \textbf{Temperature} & \textbf{Radius} \\ \hline
$\beta$ Cyg.   & 1.3($M_\odot$)   & 3.5($L_\odot$)   &                &       &    \\ \hline
$\alpha$ Cen.  & 1.0($M_\odot$)   &      &                 &       & 1($R_\odot$)   \\ \hline
$\eta$ Car.    & 60.0($M_\odot$)  & $10^6$($L_\odot$) & $8.0 \times 10^5$(years) &       &    \\ \hline
$\epsilon$ Eri.& 6.0($M_\odot$)   & $10^3$($L_\odot$) &                 & 20,000(K) &   \\ \hline
$\delta$ Scu.  & 2.0($M_\odot$)   &      & $5.0 \times 10^8$(years) &       & 2($R_\odot$)   \\ \hline
$\gamma$ Del.  & 0.7($M_\odot$)   &      & $4.5 \times 10^{10}$(years) & 5000(K)   &    \\ \hline
\end{tabular}

\end{table}


\begin{enumerate}
    \item[(a)] (4 points) Which of these stars will produce a planetary nebula.\vspace{0.5cm}  
    \item[(b)] (4 points) Elements heavier than \textit{Carbon} will be produced in which stars.\vspace{1cm}  
\end{enumerate}
 


                                                                    \end{document}